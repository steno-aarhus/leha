% Options for packages loaded elsewhere
\PassOptionsToPackage{unicode}{hyperref}
\PassOptionsToPackage{hyphens}{url}
\PassOptionsToPackage{dvipsnames,svgnames,x11names}{xcolor}
%
\documentclass[
  11pt,
  a4paper,
  DIV=11,
  numbers=noendperiod,
  twocolumn]{scrartcl}

\usepackage{amsmath,amssymb}
\usepackage{iftex}
\ifPDFTeX
  \usepackage[T1]{fontenc}
  \usepackage[utf8]{inputenc}
  \usepackage{textcomp} % provide euro and other symbols
\else % if luatex or xetex
  \usepackage{unicode-math}
  \defaultfontfeatures{Scale=MatchLowercase}
  \defaultfontfeatures[\rmfamily]{Ligatures=TeX,Scale=1}
\fi
\usepackage{lmodern}
\ifPDFTeX\else  
    % xetex/luatex font selection
\fi
% Use upquote if available, for straight quotes in verbatim environments
\IfFileExists{upquote.sty}{\usepackage{upquote}}{}
\IfFileExists{microtype.sty}{% use microtype if available
  \usepackage[]{microtype}
  \UseMicrotypeSet[protrusion]{basicmath} % disable protrusion for tt fonts
}{}
\makeatletter
\@ifundefined{KOMAClassName}{% if non-KOMA class
  \IfFileExists{parskip.sty}{%
    \usepackage{parskip}
  }{% else
    \setlength{\parindent}{0pt}
    \setlength{\parskip}{6pt plus 2pt minus 1pt}}
}{% if KOMA class
  \KOMAoptions{parskip=half}}
\makeatother
\usepackage{xcolor}
\usepackage[top=2cm,bottom=2cm,left=2cm,right=2cm]{geometry}
\setlength{\emergencystretch}{3em} % prevent overfull lines
\setcounter{secnumdepth}{-\maxdimen} % remove section numbering
% Make \paragraph and \subparagraph free-standing
\ifx\paragraph\undefined\else
  \let\oldparagraph\paragraph
  \renewcommand{\paragraph}[1]{\oldparagraph{#1}\mbox{}}
\fi
\ifx\subparagraph\undefined\else
  \let\oldsubparagraph\subparagraph
  \renewcommand{\subparagraph}[1]{\oldsubparagraph{#1}\mbox{}}
\fi


\providecommand{\tightlist}{%
  \setlength{\itemsep}{0pt}\setlength{\parskip}{0pt}}\usepackage{longtable,booktabs,array}
\usepackage{calc} % for calculating minipage widths
% Correct order of tables after \paragraph or \subparagraph
\usepackage{etoolbox}
\makeatletter
\patchcmd\longtable{\par}{\if@noskipsec\mbox{}\fi\par}{}{}
\makeatother
% Allow footnotes in longtable head/foot
\IfFileExists{footnotehyper.sty}{\usepackage{footnotehyper}}{\usepackage{footnote}}
\makesavenoteenv{longtable}
\usepackage{graphicx}
\makeatletter
\def\maxwidth{\ifdim\Gin@nat@width>\linewidth\linewidth\else\Gin@nat@width\fi}
\def\maxheight{\ifdim\Gin@nat@height>\textheight\textheight\else\Gin@nat@height\fi}
\makeatother
% Scale images if necessary, so that they will not overflow the page
% margins by default, and it is still possible to overwrite the defaults
% using explicit options in \includegraphics[width, height, ...]{}
\setkeys{Gin}{width=\maxwidth,height=\maxheight,keepaspectratio}
% Set default figure placement to htbp
\makeatletter
\def\fps@figure{htbp}
\makeatother
\newlength{\cslhangindent}
\setlength{\cslhangindent}{1.5em}
\newlength{\csllabelwidth}
\setlength{\csllabelwidth}{3em}
\newlength{\cslentryspacingunit} % times entry-spacing
\setlength{\cslentryspacingunit}{\parskip}
\newenvironment{CSLReferences}[2] % #1 hanging-ident, #2 entry spacing
 {% don't indent paragraphs
  \setlength{\parindent}{0pt}
  % turn on hanging indent if param 1 is 1
  \ifodd #1
  \let\oldpar\par
  \def\par{\hangindent=\cslhangindent\oldpar}
  \fi
  % set entry spacing
  \setlength{\parskip}{#2\cslentryspacingunit}
 }%
 {}
\usepackage{calc}
\newcommand{\CSLBlock}[1]{#1\hfill\break}
\newcommand{\CSLLeftMargin}[1]{\parbox[t]{\csllabelwidth}{#1}}
\newcommand{\CSLRightInline}[1]{\parbox[t]{\linewidth - \csllabelwidth}{#1}\break}
\newcommand{\CSLIndent}[1]{\hspace{\cslhangindent}#1}

\KOMAoption{captions}{tableheading}
\makeatletter
\makeatother
\makeatletter
\makeatother
\makeatletter
\@ifpackageloaded{caption}{}{\usepackage{caption}}
\AtBeginDocument{%
\ifdefined\contentsname
  \renewcommand*\contentsname{Table of contents}
\else
  \newcommand\contentsname{Table of contents}
\fi
\ifdefined\listfigurename
  \renewcommand*\listfigurename{List of Figures}
\else
  \newcommand\listfigurename{List of Figures}
\fi
\ifdefined\listtablename
  \renewcommand*\listtablename{List of Tables}
\else
  \newcommand\listtablename{List of Tables}
\fi
\ifdefined\figurename
  \renewcommand*\figurename{Figure}
\else
  \newcommand\figurename{Figure}
\fi
\ifdefined\tablename
  \renewcommand*\tablename{Table}
\else
  \newcommand\tablename{Table}
\fi
}
\@ifpackageloaded{float}{}{\usepackage{float}}
\floatstyle{ruled}
\@ifundefined{c@chapter}{\newfloat{codelisting}{h}{lop}}{\newfloat{codelisting}{h}{lop}[chapter]}
\floatname{codelisting}{Listing}
\newcommand*\listoflistings{\listof{codelisting}{List of Listings}}
\makeatother
\makeatletter
\@ifpackageloaded{caption}{}{\usepackage{caption}}
\@ifpackageloaded{subcaption}{}{\usepackage{subcaption}}
\makeatother
\makeatletter
\@ifpackageloaded{tcolorbox}{}{\usepackage[skins,breakable]{tcolorbox}}
\makeatother
\makeatletter
\@ifundefined{shadecolor}{\definecolor{shadecolor}{rgb}{.97, .97, .97}}
\makeatother
\makeatletter
\makeatother
\makeatletter
\makeatother
\ifLuaTeX
  \usepackage{selnolig}  % disable illegal ligatures
\fi
\IfFileExists{bookmark.sty}{\usepackage{bookmark}}{\usepackage{hyperref}}
\IfFileExists{xurl.sty}{\usepackage{xurl}}{} % add URL line breaks if available
\urlstyle{same} % disable monospaced font for URLs
\hypersetup{
  pdftitle={Protocol: Substituting red and processed meat, poultry, and fish with legumes and risk of non-alcoholic fatty liver disease in a large prospective cohort},
  pdfauthor={Fie Langmann; Daniel Borch Ibsen; Luke W. Johnston; Aurora Perez-Cornago; Christina Catherine Dahm},
  colorlinks=true,
  linkcolor={blue},
  filecolor={Maroon},
  citecolor={Blue},
  urlcolor={Blue},
  pdfcreator={LaTeX via pandoc}}

\title{Protocol: Substituting red and processed meat, poultry, and fish
with legumes and risk of non-alcoholic fatty liver disease in a large
prospective cohort}
\author{Fie Langmann \and Daniel Borch Ibsen \and Luke W.
Johnston \and Aurora Perez-Cornago \and Christina Catherine Dahm}
\date{2024-05-28}

\begin{document}
\maketitle
\ifdefined\Shaded\renewenvironment{Shaded}{\begin{tcolorbox}[boxrule=0pt, enhanced, borderline west={3pt}{0pt}{shadecolor}, breakable, interior hidden, frame hidden, sharp corners]}{\end{tcolorbox}}\fi

\hypertarget{study-information}{%
\section{Study information}\label{study-information}}

Title: Substituting red and processed meat, poultry, and fish with
legumes and risk of non-alcoholic fatty liver disease in a large
prospective cohort

\hypertarget{description}{%
\subsection{Description}\label{description}}

Western diets has been shown to cause a multitude of non-communicable
diseases. Non-alcoholic fatty liver disease (NAFLD) is the leading
chronic disease globally and strongly associated with Western dietary
patterns. Based on a combined environmental and health related focus,
legumes are increasingly being recommended as a meat substitute, and
this study therefore investigates the association between substituting
legumes for meats, poultry, and fish and the risk of developing NAFLD.

This research will use the UK Biobank Resource under Application Number
81520.

\hypertarget{objectives-and-hypotheses}{%
\subsection{Objectives and hypotheses}\label{objectives-and-hypotheses}}

\hypertarget{background}{%
\subsubsection{Background}\label{background}}

The EAT-Lancet reference diet was introduced in 2019 as a globally
sustainable and healthy diet emphasizing plant-based proteins instead of
animal-based proteins, e.g., with a recommendation of 100 g legumes
daily (1--3). Legume consumption minimizes the risk of non-alcoholic
fatty liver disease (NAFLD) in animals, by reducing build-up of fats in
the liver (4--8). NAFLD is the most prevalent chronic liver disease in
the Western countries with a prevalence of 15-45 \% (9,10). The disease
is caused by Western diets high in red meat, fats, and sugars, obesity,
physical inactivity, and smoking (9--12). As legumes are a source of
both carbohydrates and proteins, research frequently compares legumes
with other carbohydrate sources (13--17). Observed associations for
foods may be due to the specific foods themselves. However, the
associations could also represent the effect of the dietary pattern in
which the specific food is consumed. When individuals limit intake of
certain food groups, they will most often increase the intake of certain
other food groups, in an otherwise stable diet (18,19). The association
of such food intake will not only be caused by the increased consumption
of one food, but rather the substitution effect including various foods.
Replacing protein from animal sources with protein from plant sources
has previously been associated with a substantially lower mortality rate
and lower risk of NAFLD (5,20). Consumption of legumes in Western
countries has been negligible to date and the impact of markedly
increasing intakes of legumes on hepatobiliary and other diseases is
understudied. Therefore, this study aims to investigate the effect of
replacing meats, poultry, or fish with legumes and the risk of
non-alcoholic steatohepatitis (NASH) or NAFLD contingent on potential
confounding factors. As it might be more feasible for Western
populations to include legumes and substitute dietary components that
are not meats, this study will also aim to investigate the effect of a
non-specific substitution of dietary non-meat components for legumes on
the risk of NAFLD and NASH.

\textbf{Hypothesis}

\begin{itemize}
\item
  Replacing meats and poultry intakes with legumes is associated with a
  lower risk of NAFLD and NASH.
\item
  Replacing fish intake with legumes is not associated with a lower risk
  of NAFLD and NASH.
\item
  Replacing non-specific foods with legumes is not associated with a
  lower risk of NAFLD and NASH.
\end{itemize}

\hypertarget{design-plan}{%
\section{Design plan}\label{design-plan}}

\hypertarget{study-type}{%
\subsection{Study type}\label{study-type}}

\textbf{Observational Study}. Data is collected from study subjects that
are not randomly assigned to a treatment. This includes surveys, natural
experiments, and regression discontinuity designs.

\hypertarget{blinding}{%
\subsection{Blinding}\label{blinding}}

No blinding is involved in this study.

\hypertarget{study-design}{%
\subsection{Study design}\label{study-design}}

\textbf{Study population and setting}

The initial recruitment of participants for the UK Biobank started in
2006 and ran until June 2010. Of 9.2 million people identified from the
National Health Service registers and invited to participate in the
study, 5.5\% participated, approximately 500,000 participants, aged
37-73 years at baseline. The study protocol and more information are
available elsewhere (21,22). All participants gave written, informed
consent before baseline, and the study was approved by the National
Information Governance Board for Health and Social Care and the National
Health Service North West Multicentre Research Ethics Committee
(reference number 06/MRE08/65).

\hypertarget{sampling-plan}{%
\section{Sampling plan}\label{sampling-plan}}

\hypertarget{existing-data}{%
\subsection{Existing data}\label{existing-data}}

\textbf{Registration prior to analysis of the data}

All data was collected in UK Biobank prior to access and data analyses.

\hypertarget{explanation-of-existing-data}{%
\subsection{Explanation of existing
data}\label{explanation-of-existing-data}}

The UK Biobank is a large national cohort of participants in the UK,
with data collected in a standardized format the the sole purpose of
providing a data resource for researchers to use for health research.
All information about collection procedures, limitations, and sources of
bias are well established and described by the UK Biobank resource.

\hypertarget{data-collection-procedures}{%
\subsection{Data collection
procedures}\label{data-collection-procedures}}

\textbf{Study population and setting}

At baseline, participants provided detailed information on several
sociodemographic, physical, lifestyle, and health-related
characteristics via self-completed touch-screen questionnaires and a
computer assisted personal interview (23). Professionally trained staff
did physical, anthropometric, and biomedical measures following
standardized procedures (21). Diet was assessed through a touchscreen
questionnaire at baseline and a 24-hour dietary assessment tool designed
for the UK Biobank study. The 24-hour dietary assessment tool was
completed by participants up to five times and 210,965 individuals
completed one or more 24-hour dietary assessments (24).

\textbf{Assessment of diet}

The Oxford WebQ was designed as an internet based 24-hour dietary
assessment tool for measuring diet on repeated occasions. The
questionnaire is a short set of food frequency questions on commonly
eaten food groups in the British population on the day before. The
questionnaire aims to measure the type and quantity of food and
beverages consumed in the last 24 hours and estimate nutrients from the
entered information through the UK Nutrient Databank Food Composition
Tables (25,26). The Oxford WebQ was compared with
interviewer-administered 24-hour dietary recalls and validated for
macronutrients and total energy intake using recovery biomarkers and
compared with a single food frequency questionnaire (25,27,28).
Recently, the Oxford WebQ nutrient calculation was updated to provide
more detailed information on nutrient intakes and to incorporate new
dietary variables (26). Participants recruited between April 2009 and
September 2010 completed the Oxford WebQ at baseline (n=70,747). The
Oxford WebQ was not available until April 2009 and participants
recruited before that date who provided a valid email address were
invited to complete the four subsequent 24-hour dietary assessments
online (29).

\textbf{Legumes}

Legume consumption will be estimated based on participants' reported
diets from the self-administered online 24-hour dietary assessments, the
Oxford WebQ. Consumption of legumes and pulses will be based on total
weight by food group intakes estimated from participants' responses in
the Oxford WebQ. Despite the high detail level of the Oxford WebQ, a
single 24-hour dietary assessment cannot capture habitual intake of
legumes in a UK setting (30). Therefore, this study will include varying
numbers of 24-hour dietary assessments to ensure that we capture the
usual intake of legumes. Usual intake will be estimated as a daily
average across available 24-hour dietary assessments.

\textbf{Meat, poultry, and fish}

Consumption of red and processed meats, poultry, and fish will be based
on total weight by food group intakes estimated from participants'
responses to the Oxford WebQs. Red and processed meat will be defined as
beef, pork, lamb, and other meats including offal, and processed meat
including sausages, bacon, ham, and liver paté. Poultry will be defined
as poultry with or without skin, and fried poultry with batter or
breadcrumbs. Fish will be defined as oily fish, white fish and tinned
tuna, fried fish with batter or breadcrumbs, and shellfish.

\textbf{Non-alcoholic fatty liver disease}

Incident cases of NASH and NAFLD will be assessed through linkage to the
National Health Service registers where diagnosis after hospital
admission or primary care visits are coded according to the
International Classification of Diseases and Related Health Problems
(ICD-10) (31). In 2023, NAFLD underwent a change in nomenclature to
metabolic dysfunction-associated steatotic liver disease (MASLD) to
shift focus towards the metabolic factors underlying the disease and not
merely the lack of alcohol consumption and to aid the prevention and
early diagnoses of MASLD (32). However, as data was collected prior to
this change in nomenclature, cases will be classified based on the
ICD-10-codes and diagnosis criteria for NAFLD and NASH. To avoid
assuming that NAFLD and MASLD are the same, this study will define
incident cases of NAFLD based on ICD-10-code for NAFLD, K76.0, at first
admission to hospital, while incident cases of NASH are diagnosed based
on ICD-10-code, K75.8 (33,34).

\textbf{Covariates}

Directed acyclic graphs were used to illustrate the potential and known
association between covariates of the association between substituting
meat, poultry or fish for legumes and development of NAFLD and NASH.
Information on covariates will include information on all other dietary
components based on total weight by food group intakes as g/week
retrieved from the Oxford WebQ (fruits, vegetables, cereal products,
dairy products, egg products, nuts, mixed dishes, condiments, added
sugar and sweets, non-alcoholic beverages, and alcoholic beverages), sex
(male, female), age (years), alcohol consumption (g ethanol/day), ethnic
group (white, mixed or other, Asian, black), socioeconomic status
(Townsend deprivation score {[}quintiles{]}, educational level),
geographical region of recruitment (ten UK regions), cohabitation
(alone, with spouse/partner, with other non-partner, no answer),
anthropometry (BMI {[}kg/m2{]}), physical activity (low {[}≤918
MET-min/week{]}, moderate {[}918-3706 MET-min/week{]}, high {[}≥3706
MET-min/week{]}, and unknown) (35), smoking status (never, former,
current 1-15 cigarettes per day, current ≥15 cigarettes per day, and
smoking status unknown); metabolic risk factors (self-reported {[}yes or
no/unknown{]} diagnoses of diabetes, hypertension, stroke, myocardial
infarction, gallbladder disease, alcoholic liver disease or high
cholesterol); family history of metabolic risk factors (self-reported
{[}yes or no/unknown{]} diagnoses of diabetes, hypertension, stroke, or
heart disease), and cancer (yes, no, unknown).

\hypertarget{sample-size}{%
\subsection{Sample size}\label{sample-size}}

For this cohort study, only participants with two or more 24 h dietary
assessments will be included in the analyses, while missing information
on covariates will be filled in by multiple imputations where
applicable. Individuals with alcohol intakes exceeding the 90th
percentile will be excluded from the study to mitigate the risk of
misclassifying the outcome by eliminating those with alcoholic fatty
liver disease.

\hypertarget{sample-size-rationale}{%
\subsection{Sample size rationale}\label{sample-size-rationale}}

\hypertarget{variables}{%
\section{Variables}\label{variables}}

\hypertarget{measured-variables}{%
\subsection{Measured variables}\label{measured-variables}}

\textbf{Exposures:} g/week consumption of legumes, meats, poultry, and
fish

\textbf{Outcomes:} non-alcoholic fatty liver disease (NAFLD) and
non-alcoholic steatohepatitis (NASH)

\textbf{Covariates:} g/week consumption of fruits, vegetables, cereal
products, dairy products, egg products, nuts, mixed dishes, condiments,
added sugar and sweets, non-alcoholic beverages, and alcoholic
beverages. Sex, age, ethnic group, Townsend deprivation score,
educational level, geographical region of recruitment (ten UK regions),
cohabitation, BMI, physical activity, smoking status, own and family
history of metabolic risk factors, and cancer.

\hypertarget{analysis-plan}{%
\section{Analysis plan}\label{analysis-plan}}

\hypertarget{statistical-models}{%
\subsection{Statistical models}\label{statistical-models}}

\hypertarget{main-analyses}{%
\subsubsection{Main analyses}\label{main-analyses}}

Standard summary statistics will be performed to describe the
distribution of baseline characteristics and food consumption as an
average g/day based on participants' 24-h WebQ responses. Multi-variable
adjusted Cox Proportional Hazards regression models will be used to
estimate the hazard ratios for NAFLD based on replacing meats, poultry,
or fish with legumes.

\begin{itemize}
\tightlist
\item
  Replacing red and processed meats, poultry, or fish with legumes
  (e.g., per 80 g/week)
\end{itemize}

Age will be used as the underlying time scale in the analyses. Follow-up
time will begin with participants' last completed Oxford WebQ. As
participants in UKB are still followed-up today, participants will be
right censored at the date of the most recent registry update of full
follow-up for the outcomes. Otherwise, censoring will occur at the event
of death, loss to follow-up from the study, or date of diagnosis of
NAFLD or NASH, whichever comes first. The substitution analyses will be
conducted with different adjustment levels.

\textbf{Model 1} will be minimally adjusted for strata of age at
recruitment (\textgreater45, 45-49, 50-54, 55-59, 60-64, ≤65 years) and
geographical region of recruitment (ten UK regions), sex, and intake of
all other dietary components apart from the substitute components (red
and processed meats; poultry; fish). When substituting g legumes/day,
the unit for all dietary components will be g/day and the analyses will
be adjusted for total amount of consumed foods in g/day.

\textbf{Model 2} will be further adjusted for alcohol consumption
(g/day), ethnic group (white, mixed, Asian, black), socioeconomic status
(Townsend deprivation score, educational level), cohabitation (alone,
with partner, with other non-partner, unknown), physical activity (low
{[}≤918 MET-min/week{]}, moderate {[}918-3706 MET-min/week{]}, high
{[}≥3706 MET-min/week{]}, and unknown), smoking status (never, former,
current 1-15 cigarettes per day, current ≥15 cigarettes per day, and
unknown), own and family history of metabolic risk factors (yes or
no/unknown), and cancer (yes, no, unknown).

\textbf{Model 3} will further adjust for anthropometry (BMI ≥ 30 kg/m2),
as obesity may either confound or mediate the association between
replacing red and processed meats, poultry, or fish with legumes and
risk of NAFLD

\hypertarget{secondary-and-sensitivity-analyses}{%
\subsubsection{Secondary and sensitivity
analyses}\label{secondary-and-sensitivity-analyses}}

To evaluate the association between overall legume intake and NAFLD
risk, hazard ratios associated with a 80 g/week increase in legumes will
be estimated following adjustment levels in model 2. A non-specific
substitution, adjusted like model 2, will also be conducted to evaluate
the association between substituting 80 g/week of legumes for any other
dietary component.

Peas are increasingly used as a plant-based meat alternative in the food
industry (36). Despite this, peas are also included in the NHS 5 A Day
recommendations for fruit and vegetables consumption in the UK (37).
Therefore, in secondary analyses consumption of legumes will include
participants self-reported intake of legumes and pulses together with
consumed peas. The amount of peas consumed will be estimated based on
participants' reported daily portions consumed with a portion size of
peas weighing 80 g (38).

To evaluate the robustness of the main analyses, sensitivity analyses
will include varying numbers of Oxford WebQ returns, and removal of
participants with increased serum levels of alanine-aminotransferase
(\textgreater40 U/L) and aspartate-aminotransferase. Sensitivity
analyses will further include removal of participants in the highest
10\% percentile of alcohol intake.

All analyses will be conducted in R with a significance level of 5\%.

\hypertarget{inference-criteria}{%
\subsection{Inference criteria}\label{inference-criteria}}

All analyses will be evaluated based on two-sided p-values. Values below
5\% are classified as statistically significant. Inference on relevance
and significance and the evaluation of, whether a result is meaningful
will be based on the size of the estimate, and confidence intervals
containing 1 or 0, for ratios or absolute measures respectively.

\hypertarget{references}{%
\section*{References}\label{references}}
\addcontentsline{toc}{section}{References}

\hypertarget{refs}{}
\begin{CSLReferences}{0}{0}
\leavevmode\vadjust pre{\hypertarget{ref-Loken2020}{}}%
\CSLLeftMargin{1. }%
\CSLRightInline{Loken D B. Diets for a better future {[}Internet{]}. The
EAT-Lancet Commission on Food, Planet, Health; 2020. Available from:
\url{https://eatforum.org/diets-for-a-better-future-report/}}

\leavevmode\vadjust pre{\hypertarget{ref-Springmann2020}{}}%
\CSLLeftMargin{2. }%
\CSLRightInline{Springmann M, Spajic L, Clark MA, Poore J, Herforth A,
Webb P, et al. The healthiness and sustainability of national and global
food based dietary guidelines: Modelling study. BMJ {[}Internet{]}.
2020;370:m2322. Available from:
\url{https://www.bmj.com/content/bmj/370/bmj.m2322.full.pdf}}

\leavevmode\vadjust pre{\hypertarget{ref-Willett2019}{}}%
\CSLLeftMargin{3. }%
\CSLRightInline{Willett W, Rockström J, Loken B, Springmann M, Lang T,
Vermeulen S, et al.
\href{https://doi.org/10.1016/s0140-6736(18)31788-4}{Food in the
anthropocene: The EAT-lancet commission on healthy diets from
sustainable food systems}. Lancet. 2019;393(10170):447--92. }

\leavevmode\vadjust pre{\hypertarget{ref-Hong2016}{}}%
\CSLLeftMargin{4. }%
\CSLRightInline{Hong J, Kim S, Kim HS.
\href{https://doi.org/10.1089/jmf.2015.3604}{Hepatoprotective effects of
soybean embryo by enhancing adiponectin-mediated AMP-activated protein
kinase α pathway in high-fat and high-cholesterol diet-induced
nonalcoholic fatty liver disease}. J Med Food. 2016;19(6):549--59. }

\leavevmode\vadjust pre{\hypertarget{ref-Zhang2023}{}}%
\CSLLeftMargin{5. }%
\CSLRightInline{Zhang S, Yan Y, Meng G, Zhang Q, Liu L, Wu H, et al.
Protein foods from animal sources and risk of nonalcoholic fatty liver
disease in representative cohorts from north and south china. Journal of
Internal Medicine {[}Internet{]}. 2023;293(3):340--53. Available from:
\href{https://www.embase.com/search/results?subaction=viewrecord\&id=L2020249688\&from=export\%0Ahttp://dx.doi.org/10.1111/joim.13586}{https://www.embase.com/search/results?subaction=viewrecord\&id=L2020249688\&from=export
http://dx.doi.org/10.1111/joim.13586}}

\leavevmode\vadjust pre{\hypertarget{ref-Bouchenak2013}{}}%
\CSLLeftMargin{6. }%
\CSLRightInline{Bouchenak M, Lamri-Senhadji M.
\href{https://doi.org/10.1089/jmf.2011.0238}{Nutritional quality of
legumes, and their role in cardiometabolic risk prevention: A review}. J
Med Food. 2013;16(3):185--98. }

\leavevmode\vadjust pre{\hypertarget{ref-Son2014}{}}%
\CSLLeftMargin{7. }%
\CSLRightInline{Son Y, Jang MK, Jung MH.
\href{https://doi.org/10.1089/jmf.2014.3194}{Vigna nakashimae extract
prevents hepatic steatosis in obese mice fed high-fat diets}. J Med
Food. 2014;17(12):1322--31. }

\leavevmode\vadjust pre{\hypertarget{ref-Rigotti1989}{}}%
\CSLLeftMargin{8. }%
\CSLRightInline{Rigotti A, Marzolo MP, Ulloa N, González O, Nervi F.
\href{https://doi.org/10.1016/S0022-2275(20)38291-2}{Effect of bean
intake on biliary lipid secretion and on hepatic cholesterol metabolism
in the rat}. Journal of lipid research. 1989;30:1041--8. }

\leavevmode\vadjust pre{\hypertarget{ref-Benedict2017}{}}%
\CSLLeftMargin{9. }%
\CSLRightInline{Benedict M, Zhang X.
\href{https://doi.org/10.4254/wjh.v9.i16.715}{Non-alcoholic fatty liver
disease: An expanded review}. World J Hepatol. 2017;9(16):715--32. }

\leavevmode\vadjust pre{\hypertarget{ref-Younossi2016}{}}%
\CSLLeftMargin{10. }%
\CSLRightInline{Younossi ZM, Koenig AB, Abdelatif D, Fazel Y, Henry L,
Wymer M. \href{https://doi.org/10.1002/hep.28431}{Global epidemiology of
nonalcoholic fatty liver disease-meta-analytic assessment of prevalence,
incidence, and outcomes}. Hepatology. 2016;64(1):73--84. }

\leavevmode\vadjust pre{\hypertarget{ref-Younossi2018}{}}%
\CSLLeftMargin{11. }%
\CSLRightInline{Younossi ZM, Loomba R, Rinella ME, Bugianesi E,
Marchesini G, Neuschwander-Tetri BA, et al.
\href{https://doi.org/10.1002/hep.29724}{Current and future therapeutic
regimens for nonalcoholic fatty liver disease and nonalcoholic
steatohepatitis}. Hepatology. 2018;68(1):361--71. }

\leavevmode\vadjust pre{\hypertarget{ref-AlDayyat2018}{}}%
\CSLLeftMargin{12. }%
\CSLRightInline{Al-Dayyat HM, Rayyan YM, Tayyem RF.
\href{https://doi.org/10.1016/j.dsx.2018.03.016}{Non-alcoholic fatty
liver disease and associated dietary and lifestyle risk factors}.
Diabetes Metab Syndr. 2018;12(4):569--75. }

\leavevmode\vadjust pre{\hypertarget{ref-Sievenpiper2009}{}}%
\CSLLeftMargin{13. }%
\CSLRightInline{Sievenpiper JL, Kendall CW, Esfahani A, Wong JM,
Carleton AJ, Jiang HY, et al.
\href{https://doi.org/10.1007/s00125-009-1395-7}{Effect of non-oil-seed
pulses on glycaemic control: A systematic review and meta-analysis of
randomised controlled experimental trials in people with and without
diabetes}. Diabetologia. 2009;52(8):1479--95. }

\leavevmode\vadjust pre{\hypertarget{ref-Jayalath2014}{}}%
\CSLLeftMargin{14. }%
\CSLRightInline{Jayalath VH, Souza RJ de, Sievenpiper JL, Ha V,
Chiavaroli L, Mirrahimi A, et al.
\href{https://doi.org/10.1093/ajh/hpt155}{Effect of dietary pulses on
blood pressure: A systematic review and meta-analysis of controlled
feeding trials}. Am J Hypertens. 2014;27(1):56--64. }

\leavevmode\vadjust pre{\hypertarget{ref-Belski2011}{}}%
\CSLLeftMargin{15. }%
\CSLRightInline{Belski R, Mori TA, Puddey IB, Sipsas S, Woodman RJ,
Ackland TR, et al. \href{https://doi.org/10.1038/ijo.2010.213}{Effects
of lupin-enriched foods on body composition and cardiovascular disease
risk factors: A 12-month randomized controlled weight loss trial}. Int J
Obes (Lond). 2011;35(6):810--9. }

\leavevmode\vadjust pre{\hypertarget{ref-Linlawan2019}{}}%
\CSLLeftMargin{16. }%
\CSLRightInline{Linlawan S, Patcharatrakul T, Somlaw N, Gonlachanvit S.
\href{https://doi.org/10.3390/nu11092061}{Effect of rice, wheat, and
mung bean ingestion on intestinal gas production and postprandial
gastrointestinal symptoms in non-constipation irritable bowel syndrome
patients}. Nutrients. 2019;11(9). }

\leavevmode\vadjust pre{\hypertarget{ref-Lee2008}{}}%
\CSLLeftMargin{17. }%
\CSLRightInline{Lee YP, Puddey IB, Hodgson JM.
\href{https://doi.org/10.1111/j.1440-1681.2008.04899.x}{Protein, fibre
and blood pressure: Potential benefit of legumes}. Clin Exp Pharmacol
Physiol. 2008;35(4):473--6. }

\leavevmode\vadjust pre{\hypertarget{ref-Ibsen2020}{}}%
\CSLLeftMargin{18. }%
\CSLRightInline{Ibsen DB, Laursen ASD, Würtz AML, Dahm CC, Rimm EB,
Parner ET, et al. \href{https://doi.org/10.1093/ajcn/nqaa315}{Food
substitution models for nutritional epidemiology}. The American Journal
of Clinical Nutrition. 2020;113(2):294--303. }

\leavevmode\vadjust pre{\hypertarget{ref-Ibsen2022}{}}%
\CSLLeftMargin{19. }%
\CSLRightInline{Ibsen DB, Dahm CC.
\href{https://doi.org/10.1093/ajcn/nqac222}{Food substitutions
revisited}. The American Journal of Clinical Nutrition.
2022;116(5):1195--8. }

\leavevmode\vadjust pre{\hypertarget{ref-Song2016}{}}%
\CSLLeftMargin{20. }%
\CSLRightInline{Song M, Fung TT, Hu FB, Willett WC, Longo VD, Chan AT,
et al.
\href{https://doi.org/10.1001/jamainternmed.2016.4182}{Association of
animal and plant protein intake with all-cause and cause-specific
mortality}. JAMA Internal Medicine. 2016;176(10):1453--63. }

\leavevmode\vadjust pre{\hypertarget{ref-Sudlow2015}{}}%
\CSLLeftMargin{21. }%
\CSLRightInline{Sudlow C, Gallacher J, Allen N, Beral V, Burton P,
Danesh J, et al. \href{https://doi.org/10.1371/journal.pmed.1001779}{UK
biobank: An open access resource for identifying the causes of a wide
range of complex diseases of middle and old age}. PLoS Med.
2015;12(3):e1001779. }

\leavevmode\vadjust pre{\hypertarget{ref-Biobank2007}{}}%
\CSLLeftMargin{22. }%
\CSLRightInline{Biobank U. UK biobank: Protocol for a large-scale
prospective epidemiological resource {[}Internet{]}. Vol. 2021. UK
Biobank Coordinating Centre; 2007. Available from:
\url{https://www.ukbiobank.ac.uk/media/gnkeyh2q/study-rationale.pdf}}

\leavevmode\vadjust pre{\hypertarget{ref-Fry2017}{}}%
\CSLLeftMargin{23. }%
\CSLRightInline{Fry A, Littlejohns TJ, Sudlow C, Doherty N, Adamska L,
Sprosen T, et al. \href{https://doi.org/10.1093/aje/kwx246}{Comparison
of sociodemographic and health-related characteristics of UK biobank
participants with those of the general population}. Am J Epidemiol.
2017;186(9):1026--34. }

\leavevmode\vadjust pre{\hypertarget{ref-Biobank2021}{}}%
\CSLLeftMargin{24. }%
\CSLRightInline{Biobank U. Data-field 105010 {[}Internet{]}. 2021.
Available from:
\url{https://biobank.ndph.ox.ac.uk/crystal/field.cgi?id=105010}}

\leavevmode\vadjust pre{\hypertarget{ref-Liu2011}{}}%
\CSLLeftMargin{25. }%
\CSLRightInline{Liu B, Young H, Crowe FL, Benson VS, Spencer EA, Key TJ,
et al. \href{https://doi.org/10.1017/s1368980011000942}{Development and
evaluation of the oxford WebQ, a low-cost, web-based method for
assessment of previous 24 h dietary intakes in large-scale prospective
studies}. Public Health Nutr. 2011;14(11):1998--2005. }

\leavevmode\vadjust pre{\hypertarget{ref-PerezCornago2021}{}}%
\CSLLeftMargin{26. }%
\CSLRightInline{Perez-Cornago A, Pollard Z, Young H, Uden M van, Andrews
C, Piernas C, et al.
\href{https://doi.org/10.1007/s00394-021-02558-4}{Description of the
updated nutrition calculation of the oxford WebQ questionnaire and
comparison with the previous version among 207,144 participants in UK
biobank}. Eur J Nutr. 2021;60(7):4019--30. }

\leavevmode\vadjust pre{\hypertarget{ref-Greenwood2019}{}}%
\CSLLeftMargin{27. }%
\CSLRightInline{Greenwood DC, Hardie LJ, Frost GS, Alwan NA, Bradbury
KE, Carter M, et al.
\href{https://doi.org/10.1093/aje/kwz165}{Validation of the oxford WebQ
online 24-hour dietary questionnaire using biomarkers}. Am J Epidemiol.
2019;188(10):1858--67. }

\leavevmode\vadjust pre{\hypertarget{ref-Carter2019}{}}%
\CSLLeftMargin{28. }%
\CSLRightInline{Carter JL, Lewington S, Piernas C, Bradbury K, Key TJ,
Jebb SA, et al.
\href{https://doi.org/10.1017/jns.2019.31}{Reproducibility of dietary
intakes of macronutrients, specific food groups, and dietary patterns in
211 050 adults in the UK biobank study}. J Nutr Sci. 2019;8:e34. }

\leavevmode\vadjust pre{\hypertarget{ref-Kelly2021}{}}%
\CSLLeftMargin{29. }%
\CSLRightInline{Kelly RK, Watling CZ, Tong TYN, Piernas C, Carter JL,
Papier K, et al.
\href{https://doi.org/10.1161/atvbaha.120.315628}{Associations between
macronutrients from different dietary sources and serum lipids in 24 639
UK biobank study participants}. Arterioscler Thromb Vasc Biol.
2021;41(7):2190--200. }

\leavevmode\vadjust pre{\hypertarget{ref-FAO2018}{}}%
\CSLLeftMargin{30. }%
\CSLRightInline{FAO. Dietary assessment: A resource guide to method
selection and application in low ressource settings. {[}Internet{]}.
Food; Agriculture Organiszation of the United Nations; 2018. Available
from: \url{https://www.fao.org/3/i9940en/I9940EN.pdf}}

\leavevmode\vadjust pre{\hypertarget{ref-Digital2021}{}}%
\CSLLeftMargin{31. }%
\CSLRightInline{Digital N. NHS data model and dictionary international
classification of diseases (ICD) {[}Internet{]}. Vol. 2021. The National
Health Service,; 2021. Available from:
\url{https://datadictionary.nhs.uk/supporting_information/international_classification_of_diseases__icd_.html}}

\leavevmode\vadjust pre{\hypertarget{ref-Thornton2023}{}}%
\CSLLeftMargin{32. }%
\CSLRightInline{Thornton J. Associations rename fatty liver disease to
reduce stigma. BMJ {[}Internet{]}. 2023;382:p1587. Available from:
\url{http://www.bmj.com/content/382/bmj.p1587.abstract}}

\leavevmode\vadjust pre{\hypertarget{ref-Biobank2020}{}}%
\CSLLeftMargin{33. }%
\CSLRightInline{Biobank U. Hospital inpatient data {[}Internet{]}. UK
Biobank; 2020. Available from: \url{http://www.ukbiobank.ac.uk/}}

\leavevmode\vadjust pre{\hypertarget{ref-Zhang2023a}{}}%
\CSLLeftMargin{34. }%
\CSLRightInline{Zhang Z, Burrows K, Fuller H, Speliotes EK, Abeysekera
KWM, Thorne JL, et al.
\href{https://doi.org/10.3390/nu15061442}{Non-alcoholic fatty liver
disease and vitamin d in the UK biobank: A two-sample bidirectional
mendelian randomisation study}. Nutrients. 2023;15(6). }

\leavevmode\vadjust pre{\hypertarget{ref-Cassidy2016}{}}%
\CSLLeftMargin{35. }%
\CSLRightInline{Cassidy S, Chau JY, Catt M, Bauman A, Trenell MI.
Cross-sectional study of diet, physical activity, television viewing and
sleep duration in 233 110 adults from the UK biobank; the behavioural
phenotype of cardiovascular disease and type 2 diabetes. BMJ Open
{[}Internet{]}. 2016;6(3). Available from:
\url{https://bmjopen.bmj.com/content/6/3/e010038}}

\leavevmode\vadjust pre{\hypertarget{ref-Maningat2022}{}}%
\CSLLeftMargin{36. }%
\CSLRightInline{Maningat CC, Jeradechachai T, Buttshaw MR.
\href{https://doi.org/10.1002/cche.10503}{Textured wheat and pea
proteins for meat alternative applications}. Cereal Chemistry.
2022;99(1):37--66. }

\leavevmode\vadjust pre{\hypertarget{ref-Services2022}{}}%
\CSLLeftMargin{37. }%
\CSLRightInline{Services UNH. 5 a day portion sizes {[}Internet{]}. Vol.
2022. UK NHS; 2022. Available from:
\url{https://www.nhs.uk/live-well/eat-well/5-a-day/portion-sizes/}}

\leavevmode\vadjust pre{\hypertarget{ref-Services2021}{}}%
\CSLLeftMargin{38. }%
\CSLRightInline{Services UNH. 5 a day: What counts? {[}Internet{]}. Vol.
2022. UK NHS; 2021. Available from:
\url{https://www.nhs.uk/live-well/eat-well/5-a-day/5-a-day-what-counts/}}

\end{CSLReferences}



\end{document}
